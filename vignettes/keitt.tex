% !TeX root = RJwrapper.tex
\title{\pkg{odeintr}: High Performance ODE Solvers Compiled On-Demand}
\author{Timothy H. Keitt}

\maketitle

\abstract{
The odeintr package provides a simple interface for integrating systems of ordinary differential equations. Options are provide for defining the system in R-code or in C++-code. When the system is specified in C++, a set of functions is dynamically compiled and linked to the R session. These functions provide several options for integrating the system, setting and accessing state and recorded output. Flexible recording of system state, along with other information, is accomplished through defining an observer function that is called at specific times during the system integration. Options are available for equispaced, adaptive and pre-defined observer calls.
}

\section{Introduction}

Introductory section which may include references in parentheses
\citep{R}, or cite a reference such as \citet{R} in the text.

\section{Section title in sentence case}

This section may contain a figure such as Figure~\ref{figure:rlogo}.

% \begin{figure}[htbp]
%   \centering
%   \includegraphics{Rlogo}
%   \caption{The logo of R.}
%   \label{figure:rlogo}
% \end{figure}

\section{Another section}

There will likely be several sections, perhaps including code snippets, such as:

\begin{example}
  x <- 1:10
  result <- myFunction(x)
\end{example}

\section{Summary}

This file is only a basic article template. For full details of \emph{The R Journal} style and information on how to prepare your article for submission, see the \href{http://journal.r-project.org/latex/RJauthorguide.pdf}{Instructions for Authors}.

\bibliography{keitt}

\address{Timothy H. Keitt\\
  Department of Integrative Biology\\
  University of Texas at Austin\\
  USA\\}
\email{tkeitt@utexas.edu}
